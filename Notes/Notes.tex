\documentclass[]{article}

%opening
\title{Notes}

\begin{document}

\maketitle

\section{Main Idea}

Integer linear programming is pretty useful apparently. The problem boils down to solving a system of linear equations and inequalities. Just Google linear integer programming, and Wikipedia has a good example for the general type of problem that is being solved. The underlying problem is classically NP-hard, so there is an possibility to exponentially speed-up this entire set of problems with quantum annealing. The main idea here is to encode inequalities into a quantum annealer.

Quick overview, a quantum annealers solves a QUBO problem (which is basically Ising Model). The problem can be boiled down to the form $x Q x$ where $x$ is a vector of qubits (so they can only take values of 0 or 1 or some super-position). We have to ``program'' $Q$ to solve our problem.

Following https://arxiv.org/abs/1812.06917 we know 1) how to encode fixed-precision numbers into a set of qubits. 2) We also know how to encode higher order problems (\textit{e.g.} $J x^4 + Q x^2$). 3)  We also learned that in general, since QA can only solve for the global minimum, we must optimize the squared-residual of a set of equations / inequalities. Anyways, read the paper if you haven't already.

To encode an inequality of the form $A < y < B$ we can introduce an auxiliary qubit $y_a$ such that the logical qubit is $Y = (y, y_a)$. The inequality is then encoded as minimizing the following $C_1 Y^4 - C_2 Y^2 + C_3$ such that we get a mexican-hat potential. Basically this was inspired from spontaneous symmetry breaking. I'm sure this can be connected to solving scalar phi-4 theory if we reallllly want to go there. But tuning the $C$s will tune the radius of the degeneracy and therefore change the range of $A$ and $B$.  Also $C_3$ shifts the minimum to zero, which is useful if we want to keep the energy positive, and interpret it as the squared-residual. Travis suggested that this might be connected to ``slack variables'' (see https://arxiv.org/abs/1811.11538 that I haven't read yet).

I'm worried that $y_a$ has nonlinear dependence with $y$ as we walk along the degeneracy though, which in practice may be an issue. Though increasing the number of qubits used to represent a fixed-precision decimal solves this problem exponentially quickly.

\end{document}
