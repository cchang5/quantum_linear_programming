\documentclass[fleqn,10pt]{wlscirep}
\usepackage[utf8]{inputenc}
\usepackage[T1]{fontenc}
\usepackage{amsmath,amssymb}
\usepackage{bm}
\usepackage{graphicx}
\usepackage{color}

\usepackage{hyperref}
\hypersetup{
    colorlinks=true,       % false: boxed links; true: colored links
    linkcolor=blue,          % color of internal links
    citecolor=blue,        % color of links to bibliography
    filecolor=blue,      % color of file links
    urlcolor=blue           % color of external links
}

\title{Solutions to Integer Linear Programming from Quantum Annealing}

\author[1,*]{Chia~Cheng~Chang}
\author[2]{Chih-Chieh~Chen}
\author[3]{Christopher K\"orber}
\author[4]{Travis~Humble}
\author[5]{Jim~Ostrowski}

\affil[1]{RIKEN Interdisciplinary Theoretical and Mathematical Sciences (iTHEMS), Wako, Saitama 351-0198, Japan}
\affil[1]{Department of Physics, University of California, Berkeley, California 94720, USA}
\affil[1]{Nuclear Science Division, Lawrence Berkeley National Laboratory, Berkeley, California 94720, USA}
\affil[2]{?}
\affil[3]{Department of Physics, University of California, Berkeley, California 94720, USA}
\affil[4]{Quantum Computing Institute, Oak Ridge National Laboratory, Oak Ridge, Tennessee 37831, USA}
\affil[4]{University of Tennessee}
\affil[*]{chiacheng.chang@riken.jp}

\begin{abstract}
\cite{Chang:2018uxx}
\end{abstract}

\begin{document}

\flushbottom
\maketitle

\noindent\textbf{INTRODUCTION.}

Motivate ILP and quantum annealing.

\noindent{\textit{\textbf{Integer Linear Programming.}}}

Problem definition and classical solutions.

\noindent{\textbf{RESULTS}}

\noindent\textit{\textbf{ILP Mapping to QUBO}}

Write down how the mapping is done.

\noindent\textit{\textbf{Application to Minimum Dominating Set}}

Write down mapping to QUBO

\noindent{\textbf{DISCUSSION}}

\noindent\textit{\textbf{ILP examples}}

Simple problems and scaling with problem size for single global solution.

- number of constraint equations (slack variables)
- range of integer values
- number of variables
- precision of problem (decimal places needed)

Problems with degenerate minima

- does D-Wave find them with expected distribution for very small systems?
- at what problem size does this break?

\noindent\textit{\textbf{Dominating Set}}

Demonstrate simple dominating set problem.

Study scaling with random graphs (but fixed number of vertices?)

\noindent\textbf{METHODS}

Might not be necessary to have this section.


\noindent{\textbf{DATA AVAILABILITY}
	
Link to GitHub.

\noindent{\textbf{ACKNOWLEDGMENTS}

We thank x, y, z...

\noindent{\textbf{AUTHOR CONTRIBUTIONS}}

...

\noindent{\textbf{ADDITIONAL INFORMATION}}

\textbf{Competing Interests:} The authors declare no competing interests.

\bibliography{qilp}

\end{document}
